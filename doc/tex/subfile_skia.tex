
\section{Skia}
\label{sec:skia}
Skia的主要任务,是处理文字、几何图形、图片等的2D绘画工作,包括字体反锯齿、透明处理、特效处理等等。图片解码、PDF解析、OpenGL支持等,都不是它的主业,而只是它的副业,它也是调用开源的解码库来完成操作。

\subsection{输入输出}
\label{sec:skia-input-output}

\subsection{核心模块}
\label{sec:skia-core}
Android的上层应用,画图、打印文字、画面的特效、缩放、旋转、变换,最终都是在Skia层进行数据处理,这些操作也是Skia的核心功能。
\subsubsection{SkCanvas}
\label{sec:skia-skcanvas}
这个类是Skia引擎的一个核心类,他封装了所有对设备进行的画图操作,所有的绘画动作最终都会调用此类。这个类自身包含了一个设备的引用,以及一个矩阵和裁剪栈。所有的画图操作,都是在经过栈内存放的矩阵变幻之后才进行的(这点和OpenGL类似)。当然,最终显示给用户的图像,还必须经过裁剪堆栈的运算。

SkCanvas记录着整个设备的绘画状态,而设备上面绘制的对象的状态又是由SkPaint类来记录的,SkPaint类作为参数,传递给不同SkCanvas类的成员函数drawXXXX().(比如:drawPoints, drawLine, drawRect, drawCircle)。SkPaint类里记录着如颜色(color), 字体(typeface), 文字大小(textSize), 文字粗细(strokeWidth), 渐变(gradients, patterns)等。

这个类,主要是给JNI层的graphics和webkit层提交接口,此类的函数,主要分为以下几类:
\begin{enumerate}
\item 构造函数析构函数,给定一个Bitmap或者Device,在给定的这个对象上进行画图,Device可以为空。
\item 保存信息和恢复相关,用于保存和恢复显示矩阵,剪切,过滤堆栈,不同函数有不同的附加功能:
  \begin{itemize}
  \item virtual int \underline{\textit{save}}(SaveFlags flags);//这个函数是用来保存矩形和剪裁的信息,然后将一个副本保存在一个私有栈中,后续对他的像旋转,缩放均在这个副本中进行,当调用复原函数的时候这个副本被删除。
  \item virtual int \textit{saveLayer}(const SkRect* bounds, const SkPaint* paint, aveFlags flags);
  \item int \textit{saveLayerAlpha}(const SkRect* bounds, U8CPU alpha, SaveFlags flags);
  \item void \textit{restore}();
  \item int \textit{getSaveCount}() const;//在栈中保存的状态数
  \end{itemize}
\item 跟变换相关的,移位,缩放,旋转,变形函数:
  \begin{itemize}
  \item virtual bool \textit{translate}(SkScalar dx,SkScalar dy);
  \item virtual bool \textit{scale}(SkScalar sx, SkScalar sy);
  \item virtual bool \textit{rotate}(SkScalar degrees);
  \item virtual bool \textit{skew}(SkScalar sx, SkScalar sy);
  \item virtual bool \textit{concat}(const SkMatrix\& matrix);
  \item virtual void \textit{setMatrix}(const SkMatrix\& matrix);
  \item void \textit{resetMatrix}();
  \end{itemize}
\item 绘制图画文本的函数:
  \begin{itemize}
  \item void \textit{drawARGB}(U8CPU a, U8CPU r, U8CPU g, U8CPU b, SkXfermode::Mode mode = SkXfermode::kSrcOver\_Mode);
  \item void \textit{drawColor}(SkColor c, SkXfermode::Mode mode = SkXfermode::kSrcOver\_Mode);
  \item void \textit{drawPaint}(const SkPaint\& paint);
  \item virtual void \textit{drawPoints}(PointMode mode, size\_t count, const SkPoint pts[], const SkPaint\& paint);
  \item void \textit{drawPoint}(SkScalarx, SkScalar y, const SkPaint\& paint);
  \item void \textit{drawPoint}(SkScalarx, SkScalar y, SkColor color);
  \item void \textit{drawLine}(SkScalarx0, SkScalar y0, SkScalar x1, SkScalar y1, const SkPaint\& paint);
  \item virtual void \textit{drawRect}(const SkRect\& rect, const SkPaint\& paint);
  \item void \textit{drawIRect}(const SkIRect\& rect, const SkPaint\& paint);
  \item void \textit{drawRectCoords}(SkScalar left, SkScalar top, SkScalar right, SkScalar bottom, const SkPaint\& paint);
  \item void \textit{drawOval}(const SkRect\& oval, const SkPaint\&);
  \item void \textit{drawCircle}(SkScalar cx, SkScalar cy, SkScalar radius, const SkPaint\& paint);
  \item void \textit{drawArc}(const SkRect\& oval, SkScalar startAngle, SkScalar sweepAngle, bool useCenter, const SkPaint\& paint);
  \item void \textit{drawRoundRect}(const SkRect\& rect, SkScalar rx, SkScalar ry, const SkPaint\&paint);
  \item virtual void \textit{drawPath}(const SkPath\& path, const SkPaint\& paint);
  \item virtual void \textit{drawBitmap}(const SkBitmap\& bitmap, SkScalar left, SkScalar top, const SkPaint*paint = NULL);
  \item virtual void \textit{drawBitmapRect}(const SkBitmap\& bitmap, const SkIRect* src, const SkRect\& dst, const SkPaint* paint = NULL);
  \item virtual void \textit{drawBitmapMatrix}(const SkBitmap\& bitmap, const SkMatrix\& m, const SkPaint* paint = NULL);
  \item virtual void \textit{drawSprite}(const SkBitmap\& bitmap, int left, int top, const SkPaint* paint = NULL);
  \item virtual void \textit{drawText}(const void* text, size\_t byteLength, SkScalar x, SkScalar y, const SkPaint\& paint);
  \item virtual void \textit{drawPosText}(const void* text, size\_t byteLength, const SkPointpos[], const SkPaint\& paint);
  \item virtual void \textit{drawPosTextH}(const void* text, size\_t byteLength, const SkScalarxpos[], SkScalar constY, const SkPaint\& paint);
  \item void \textit{drawTextOnPathHV}(const void* text, size\_t byteLength, const SkPath\&path, SkScalar hOffset, SkScalar vOffset,const SkPaint\& paint);
  \item virtual void \textit{drawTextOnPath}(const void* text, size\_t byteLength, const SkPath\& path, const SkMatrix* matrix, const SkPaint\& paint);
  \item virtual void \textit{drawPicture}(SkPicture\& picture);
  \item virtual void \textit{drawShape}(SkShape*);
  \item virtual void \textit{drawVertices}(VertexMode vmode, int vertexCount, const SkPoint vertices[], const SkPoint texs[], const SkColor colors[], SkXfermode* xmode, const uint16\_t indices[], int indexCount, constSkPaint\& paint);
  \end{itemize}

\end{enumerate}

SkCanvas中有一个私有类成员:\verb|MCRec* fMCRec|,这个类是在堆栈中保存每一个 save/restore level 的记录。一个level 可选地复制 matrix 和/或 stack,我们有这些matrix / stack的指针。如果一个level 的matrix 或者stack 被拷贝,他的副本被保存在存储区域,并由指针指向他,如果没有被拷贝,我们就忽略他,指针指向栈中以前 level 的相应值。

\subsubsection{SkBitmap}
\label{sec:skia-skbitmap}
SkBitmap是光栅位图。有整数的宽高和格式。可以直接访问像素。能够用来做SkCanvas画布。它的内部有自己的一些逻辑,尤其是使用了参考计数器,可以避免多重copy,同样增加了锁机制。位图的空间可以由外部给定。

\begin{enumerate}
\item 机制相关
  \begin{itemize}
  \item bool \textit{allocPixels}(SkColorTable* ctable = NULL); //这两个函数很重要,创建位图数据参考计数器。因为只有有了参考计数器,才能对位图的多重引用产生作用。这里也会分配位图的内存空间。空间的大小是当前图像配置的大小。这里同样可以看到,对位图数据访问时需要lock和unlock。
  \item bool \textit{allocPixels}(Allocator* allocator, SkColorTable* ctable);
  \item void \textit{lockPixels}() const; //这两个函数都是成对出现,在访问SkBitmap像素的数据之前之后要分别调用,进行加锁和解锁操作。
  \item void \textit{unlockPixels}() const;
  \item void \textit{notifyPixelsChanged}() const; //通知图像有改变
  \end{itemize}

\item 操作接口
  \begin{itemize}
  \item void \textit{eraseColor}(SkColor c) const; //用指定的颜色擦除
  \item bool \textit{scrollRect}(const SkIRect* subset, int dx, int dy, SkRegion* inval = NULL) const;
  \item bool \textit{copyPixelsFrom}(const void* const src, size\_t srcSize, int srcRowBytes = -1);
  \item bool \textit{copyPixelsTo}(void* const dst, size\_t dstSize, int dstRowBytes = -1) const;
  \item bool \textit{copyTo}(SkBitmap* dst, Config c, Allocator* allocator = NULL) const;
  \item void \textit{flatten}(SkFlattenableWriteBuffer\&) const;
  \item void \textit{unflatten}(SkFlattenableReadBuffer\&);
  \end{itemize}

\item 获取信息
  \begin{itemize}
  \item int \textit{width}() const { return fWidth; }
  \item int \textit{height}() const { return fHeight; }
  \item Config \textit{getConfig}() const { return this->config(); } 
  \item int \textit{rowBytes}() const { return fRowBytes; } //根据位图的宽、高、格式,获取每行的字节数
  \item int \textit{bytesPerPixel}() const { return fBytesPerPixel; } //每像素的字节数
  \item int \textit{rowBytesAsPixels}() const { return fRowBytes >> (fBytesPerPixel >> 1); } //一行的像素数
  \item void* \textit{getPixels}() const { return fPixels; } //位图数据的起始地址
  \item size\_t \textit{getSize}() const { return fHeight * fRowBytes; } //获得像素的字节空间大小,如果大小超过32位(即4294967296)则会截断,也就是只取最后的32位。
  \item bool \textit{isOpaque}() const; //是否不透明的
  \item bool \textit{readyToDraw}() const; //确保当前SkBitmap可以用来绘图了
  \item void* \textit{getAddr}(int x, int y) const; //获得指定点的地址
  \end{itemize}
\end{enumerate}

\subparagraph{内存管理}
\label{sec:skbitmap-mm}

Allocator是SkBitmap的内嵌类,其实只有一个成员函数:allocPixelRef(),SkBitmap使用Allocator的派生类--HeapAllocator作为它的默认分配器。

Allocator继承自SkPixelRef,Allocator分配的内存由SkPixelRef来处理引用计数,每个Allocator对应一个SkPixelRef,通常在分配内存成功后,由Allocator调用setPixelRef来进行绑定。默认的情况下,SkBitmap使用SkMallocPixelRef和HeapAllocator进行配对。所以如果你要派生Allocator类,通常也需要派生一个SkPixelRef类与之对应。

\subsubsection{其他辅助类}
\label{sec:skia-others}
\begin{description}
  \item[SkDevice] SkDevice是更底层的类,为SkCanvas类提供支持,SkCanvas类里的方法,最终调用的是SkDevice类里的方法。基于SkDevice类之上的,有Skia模块、webkit核心这两大基础库。
  \item[SkDraw] SkDraw是比SkDevice更底层的类,为SkDevice层提供接口。这里面调用了更底层的类,比如SkScan扫描线填充算法\footnote{\url{http://en.wikipedia.org/wiki/Flood\_fill}}。如果想看看更底层的接口,可以看看SkBlitter类,对每个点进行的操作在这里完成。
  \item[SkPaint] 画笔类,主要作用是设置画笔的颜色、字体信息、绘画方式,用来画文字、图片、几何图形等。
  \item[SkColor] 颜色类,颜色各信息的获取接口和颜色的转换工具。
  \item[SkPicture] 记录绘画命令并且稍后播放。相关的类有SkPictureRecord和SkPicturePlayback。
  \item[SkCordic] 三角函数相关的类。
  \item[SkPath] 用于保存几何图形。
\end{description}

\subsection{图形解码}
\label{sec:skia-image}
图形解码器,采用插件的形式,按需要可以添加新的解码器进去,只需要注册一下,然后复写onDecode方法即可。

\subsubsection{资源解码过程}
\label{sec:skia-decode-res}

\begin{enumerate}
\item 上层调用BitmapFactory的接口:mFillBitmap = BitmapFactory.decodeResource(resources, R.\-drawable.\-trackingview\_shot);
\item BitmapFactory里调用bm = decodeResourceStream(res, value, is, null, opts)函数
\item 这个函数里面,return decodeStream(is, pad, opts);
\item 调用JNI层:bm = nativeDecodeStream(is, tempStorage, outPadding, opts);
\item 这里面继续调用JNI层的doDecode函数:bitmap = doDecode(env, stream, padding, options, false);
\item JNI层获取解码器:SkImageDecoder* decoder = SkImageDecoder::Factory(stream);
\item 然后调用Skia层的SkImageDecoder::decode()函数,然后调用各解码器自己的onDecode()函数来完成解码。
\end{enumerate}

\subsubsection{BMP详细解码}
\label{sec:bmp-decode}

\subparagraph{入口:}
SkBMPImageDecoder类的onDecode方法。

\subparagraph{传入参数:}
  \begin{itemize}
  \item SkStream* stream:输入流
  \item SkBitmap* bm:解码后的存储地址
  \item Mode mode:只返回bitmap的width/height/config信息,还是包括每个像素信息
  \end{itemize}

\subparagraph{返回值:}
是否解码成功

\subparagraph{过程:}
\begin{enumerate}
\item 获取stream的length
\item 限制数据流不大于6*1024*1024
\item 分配一个length大小的内存空间,拷贝stream数据进空间
\item 开始解码,调用 BmpDecoderHelper类的DecodeImage方法,设置width*height 最大值为 16383*16383,解码刚才拷贝进去的数据到BmpDecoderHelper的[r,g,b]数组中去
\item 释放刚才分配的空间
\item 调用SkScaledBitmapSampler类的begin和next方法,按照采样率对像素取样,根据config转化[r,g,b]成对应格式的数据,然后存储进bitmap里
\end{enumerate}

\subsubsection{上层区域解码流程}
\label{sec:skia-region-from-app}
以oppo/app/multimedia/Gallery2/src/com/android/gallery3d/app/CropImage.java为例,检测是否LARGE BITMAP,如果是就发送消息:
\begin{lstlisting}[language={[ANSI]C}]
mMainHandler.sendMessage(mMainHandler.obtainMessage(MSG_LARGE_BITMAP, decoder));
\end{lstlisting}
Activity收到消息后,把消息体传给区域解码接口:
\begin{lstlisting}[language={[ANSI]C}]
case MSG_LARGE_BITMAP: 
    mProgressDialog.dismiss();
    onBitmapRegionDecoderAvailable((BitmapRegionDecoder) message.obj);
\end{lstlisting}
区域解码检测width、height等信息,设置sampler等信息,然后调用BitmapRegionDecoder::decodeRegion方法:
\begin{lstlisting}[language={[ANSI]C}]
mBitmap = regionDecoder.decodeRegion(new Rect(0, 0, width, height), options);
\end{lstlisting}
此方法里调用nativeDecodeRegion接口,将区域解码信息传过去:
\begin{lstlisting}[language={[ANSI]C}]
return nativeDecodeRegion(mNativeBitmapRegionDecoder, rect.left, rect.top, rect.right - rect.left, rect.bottom - rect.top, options);
\end{lstlisting}
JNI里的此接口,调用SkBitmapRegionDecoder::decodeRegion函数,从底层解码获取到的数据存储在新建的bitmap里面,然后以此数据GraphicsJNI::createBitmap:
\begin{lstlisting}[language={[ANSI]C}]
static jobject nativeDecodeRegion(JNIEnv* env, jobject,SkBitmapRegionDecoder *brd, int start_x, int start_y, int width, int height, jobject options)
    ......
if (!brd->decodeRegion(bitmap, region, prefConfig, sampleSize)) {
        return nullObjectReturn("decoder->decodeRegion returned false");
    }
\end{lstlisting}
这里面调用SkImageDecoder::decodeRegion函数:
\begin{lstlisting}[language={[ANSI]C}]
return fDecoder->decodeRegion(bitmap, rect, pref);
\end{lstlisting}
然后调用各解码器自己的onDecodeRegion函数:
\begin{lstlisting}[language={[ANSI]C}]
if (!this->onDecodeRegion(&tmp, rect)) {
    return false;
}
\end{lstlisting}

\subsubsection{JPEG底层区域解码}
\label{sec:jpeg-region-decode}

\subparagraph{入口:}
SkJPEGImageDecoder::onDecodeRegion(SkBitmap* bm, SkIRect region)

\subparagraph{传入参数:}
  \begin{enumerate}
  \item SkBitmap* bm: 保存解码后的数据
  \item SkIRect region: 解码的区域
  \end{enumerate}

\subparagraph{流程:}
  \begin{enumerate}
  \item 新建一个SkBitmap类型临时对象bitmap
  \item 如果缩放比率为1,且格式为ARGB\_8888或RGB\_565,设置config后软解,裁剪bitmap,最终在canvas上drawBitmap后退出
  \item 如果不满足上述条件,根据采样率设置bitmap的config,先硬解码,如果不成功则软解码
  \end{enumerate}

\subparagraph{什么时候不用硬解码:}
  \begin{enumerate}
  \item 解码区域宽高之积大于25*1024*1024
  \item 采样率为1,且是android直接支持的两种格式
  \item 硬解码不成功
  \end{enumerate}

\subparagraph{硬解过程:}
  \begin{enumerate}
  \item 一行一行读出数据写入硬解码buffer
  \item 调用MDPResizer解码buffer中数据
  \item 最终数据存入bitmap中
  \end{enumerate}

\subparagraph{软解过程:}
  \begin{enumerate}
  \item 如果采样率为1,则通过函数jpeg\_read\_tile\_scanline一行一行读出数据
  \item 采样率非1则一个一个像素读出数据
  \item 最终数据存入bitmap中
  \end{enumerate}

最终,都是通过cropBitmap将临时bitmap中的数据裁剪后存入传入的bm地址并画出。

\subsection{PDF解码}
\label{sec:skia-pdf}

\subsection{SVG解码}
\label{sec:skia-svg}

\subsection{GPU相关}
\label{sec:skia-gpu}
SkGpuDevice类,只被webkit底层和skia bench/gm 这两个程序使用,并未给其他应用提供接口。SkGpuCanvas类,只供Skia bench/gm这两个内部程序使用。所以,Skia并未提供硬件加速?

根据\href{http://dev.dorothybrowser.com/?p=41}{这篇文章}的实验,Skia GPU处理速度比CPU处理慢这么多?
\subsection{特效处理}
\label{sec:skia-effect}
Skia中已经实现了以下特效:遮罩、浮雕、模糊、滤镜、渐变色、离散、透明等。但是JNI层,只提供了浮雕和模糊两种特效的接口给上方。

\subsection{调试技巧}
\label{sec:skia-debug}

\subsubsection{保存区域解码图片}
\label{sec:skia-debug-save-image}

在上层保存区域解码出来的图片,可以在BitmapRegionDecoder.java文件里打入这个patch:
\begin{lstlisting}[language={Java},label=skia-debug-save-image,caption=补丁]
 import java.io.FileInputStream;
 import java.io.IOException;
 import java.io.InputStream;
+import java.io.OutputStream;
+import java.io.FileOutputStream;
+import android.util.Log;
+import java.lang.Throwable;
 
 /**
  * BitmapRegionDecoder can be used to decode a rectangle region from an image.
@@ -35,6 +39,7 @@
  */
 public final class BitmapRegionDecoder {
     private int mNativeBitmapRegionDecoder;
+    private static int photonumber;
     private boolean mRecycled;
 
     /**
@@ -180,11 +185,35 @@
      */
     public Bitmap decodeRegion(Rect rect, BitmapFactory.Options options) {
         checkRecycled("decodeRegion called on recycled region decoder");
+        Log.d("Daclean", "decodeRegion rect.left="+rect.left+" decodeRegion rect.top="+rect.top +" decodeRegion rect.right="+rect.right+" decodeRegion rect.bottom="+rect.bottom, new Throwable("decodeRegion"));
         if (rect.left < 0 || rect.top < 0 || rect.right > getWidth()
                 || rect.bottom > getHeight())
             throw new IllegalArgumentException("rectangle is not inside the image");
-        return nativeDecodeRegion(mNativeBitmapRegionDecoder, rect.left, rect.top, rect.right - rect.left, rect.bottom - rect.top, options);
+
+        Bitmap saveBitmap;
+        saveBitmap = nativeDecodeRegion(mNativeBitmapRegionDecoder, rect.left, rect.top, rect.right - rect.left, rect.bottom - rect.top, options);
+            
+        String pathName = "/mnt/sdcard/Photo/aa"+photonumber+".png";
+        photonumber++;
+        OutputStream stream = null;
+        try {
+            stream = new FileOutputStream(pathName);         
+            saveBitmap.compress(Bitmap.CompressFormat.PNG,85,stream);
+        } catch (Exception e ) {
+        } finally{
+            if(stream != null ){
+                try{
+                    stream.close();
+                } catch(IOException e) {
+                }
+            }
+        }
+         
+        return saveBitmap; 
     }
 
     /** Returns the original image's width */
\end{lstlisting}

\subsubsection{获取解码数据}
\label{sec:skia-debug-raw}

在Skia底层保存原始数据,MTK在SkImageDecoder\_libjpeg.cpp中提供了一个保存SkBitmap的raw数据的接口store\_raw\_data,可以直接调用。
\begin{lstlisting}[language={[ANSI]C},label=skia-deug-raw,caption=获取SkBitmap数据]
int index_file = 0;
bool store_raw_data(SkBitmap* bm)
{
    FILE *fp;
    char name[150];
    unsigned long u4PQOpt;
    char value[PROPERTY_VALUE_MAX];
    //property_get("decode.hw.dump", value, "0");

    u4PQOpt = atol(value);

    //if( u4PQOpt == 0) return false;

    if(bm->getConfig() == SkBitmap::kARGB_8888_Config)
        sprintf(name, "/sdcard/dump/dump_%d_%d_%d.888", bm->width(), bm->height(), index_file++);
    else if(bm->getConfig() == SkBitmap::kRGB_565_Config)
        sprintf(name, "/sdcard/dump/dump_%d_%d_%d.565", bm->width(), bm->height(), index_file++);
    else
        return false;

    SkDebugf("store file : %s ", name);

    fp = fopen(name, "wb");
    if(fp == NULL)
    {
        SkDebugf(" open file error ");
        return false;
    }
    if(bm->getConfig() == SkBitmap::kRGB_565_Config)
    {
        fwrite(bm->getPixels(), 1 , bm->getSize(), fp);
        fclose(fp);
        return true;
    }

    unsigned char* addr = (unsigned char*)bm->getPixels();
    SkDebugf("bitmap addr : 0x%x, size : %d ", addr, bm->getSize());
    for(int i = 0 ; i < bm->getSize() ; i += 4)
    {
        fprintf(fp, "%c", addr[i]);
        fprintf(fp, "%c", addr[i+1]);
        fprintf(fp, "%c", addr[i+2]);
    }
    fclose(fp);
    return true;
}

\end{lstlisting}

\subsection{好的设计}
\label{sec:skia-design}

\subsubsection{工厂模式}
\label{sec:skia-factory}

开机后,会把所有的工厂注册到一个链表上去。当需要解码器的时候,搜索链表,搜索到满足解码器要求的工厂,就由此工厂创建一个此解码器的对象。好处是,降低了耦合度,每个解码器都可以作为一个单独的插件,即插即用。

\subparagraph{如何注册工厂?}

举个例子,这是注册WBMP文件解码器的代码:
\begin{lstlisting}[language={[ANSI]C},label=skia-reg-decoder,caption=注册解码器范例]
#include "SkTRegistry.h"

static SkImageDecoder* Factory(SkStream* stream) {
    wbmp_head   head;

    if (head.init(stream)) {
        return SkNEW(SkWBMPImageDecoder);
    }
    return NULL;
}

static SkTRegistry<SkImageDecoder*, SkStream*> gReg(Factory);
\end{lstlisting}

我们先创建一个工厂Factory,然后实例化SkTRegistry模板,SkTRegistry的构造函数中,会将此工厂链接进链表,于是就完成了注册工厂到链表里的过程。

\subparagraph{如何搜索工厂?}

搜索工厂之前,需要先将资源解码成数据流,关于资源解码过程,可以参考\ref{sec:skia-decode-res}节,搜索工厂,是在下面函数里完成的。

\begin{lstlisting}[language={[ANSI]C},label=skia-search-factory,caption=搜索工厂]
SkImageDecoder* SkImageDecoder::Factory(SkStream* stream) {
    SkImageDecoder* codec = NULL;
    const DecodeReg* curr = DecodeReg::Head();
    while (curr) {
        codec = curr->factory()(stream);
        // we rewind here, because we promise later when we call ``decode'', that
        // the stream will be at its beginning.
        stream->rewind();
        if (codec) {
            return codec;
        }
        curr = curr->next();
    }
#ifdef SK_ENABLE_LIBPNG
    codec = sk_libpng_dfactory(stream);
    stream->rewind();
    if (codec) {
        return codec;
    }
#endif
    return NULL;
}
\end{lstlisting}

先是将指针指向工厂链表的头,然后拿着样品(SkStream* stream),一个工厂一个工厂的询问,如果有工厂满足需求,则在此工厂创建一个解码器对象,搜索工厂工作就完成了。

\subsubsection{引用计数}
\label{sec:skia-ref-count}



%%% Local Variables: 
%%% mode: latex
%%% TeX-master: t
%%% End: 
