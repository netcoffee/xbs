\section{git相关}

参考手册:\url{http://git-scm.com/book/zh}

\subsection{安装配置}
\label{sec:git-install-config}

\paragraph{安装}
\label{sec:git-install}

\begin{description}
\item[Linux] 命令:
\begin{lstlisting}[numbers=none]
# apt-get install git
\end{lstlisting}
\item[Windows] 在\url{http://code.google.com/p/msysgit/}下载,有GUI和命令行两种模式。
\item[Mac] 暂空
\end{description}

\paragraph{配置}
\label{sec:git-config}

创建一个指向版本管理里的git配置文件的符号链接即可:
\begin{lstlisting}[numbers=none]
# ln -s /work/git/xbs/git/gitconfig ~/.gitconfig
\end{lstlisting}

\begin{description}
\item[忽略文件] 将文件名或者文件类型添加到项目根目录下的.gitignore文件中,支持正则表达式,然后提交到库里面去,其他人就可以共用此设置了。
\end{description}



\subsection{日常操作}
\label{sec:git-usage}

\begin{description}
\item[创建存储] 推荐在\url{https://bitbucket.org}上创建存储空间,因为在这里,可以不公开自己的项目代码,也能使用git工具。
\item[获取代码] 例如,获取我的一个项目代码,会直接在当前目录下新建一个目录,里面有.git版本控制信息。
\begin{lstlisting}[numbers=none]
# git clone https://netcoffee@bitbucket.org/netcoffee/xbs.git
\end{lstlisting}

\item[查看log] 按照\ref{sec:git-config}里的“增强log显示”配置后,\verb| git lg |可显示优化后的log信息。
\item[添加文件] 命令:
\begin{lstlisting}[numbers=none]
# touch README
# git add .
\end{lstlisting}
\item[撤销添加] 命令:
\begin{lstlisting}[numbers=none]
# git reset HEAD README
\end{lstlisting}
\item[撤销修改] 命令:
\begin{lstlisting}[numbers=none]
# git checkout -- foobar.c
\end{lstlisting}
\item[提交修改] 命令:
\begin{lstlisting}[numbers=none]
# git commit .
\end{lstlisting}

如果是在emacs中,修改保存后,使用\verb|C-x v v|命令,然后写好注释,\verb|C-c C-c|即可提交,但是似乎只能对单个文件?

如果提交后发现遗漏了一个文件,可以使用\verb|--amend|参数重新提交一遍:
\begin{lstlisting}[numbers=none]
# git commit -m 'initial commit'
# git add forgotten_file
# git commit --amend
\end{lstlisting}
\item[上传修改] 
如果是第一次上传,
\begin{lstlisting}[numbers=none]
# git push origin master
\end{lstlisting}
后续可以直接上传:
\begin{lstlisting}[numbers=none]
# git push -u
\end{lstlisting}

\item[查看修改] 比如查看最近两次修改详细信息:
\begin{lstlisting}[numbers=none]
# git log -p -2
\end{lstlisting}

\end{description}

