\section{Emacs相关}
\subsection{安装配置}
Emacs的配置文件,主要是一个启动文件,和一个存放第三方工具的目录,只需要把这两个文件,创建符号链接到相应的目录下即可,符号链接的目的地,不可更改,因为init.el里面写死了第三方的地址为\~/.emacs.d/site-lisp/目录。
\begin{lstlisting}[numbers=none]
# ln -s /work/git/xbs/emacs/init.el ~/.emacs.d/init.el
# ln -s /work/git/xbs/emacs/site-lisp/ ~/.emacs.d/site-lisp
\end{lstlisting}

\subsection{常用操作}

\subparagraph{终端下选择菜单:}
可以先尝试F10,在Putty下面是可以工作的,如果不行的话,用命令M-x menu-bar-open也可以选择。
\subparagraph{跳转行数:}
\verb|M-G M-G|
\subparagraph{合并此行到上一行:}
\verb|M-^|
\subparagraph{undo/redo:}
安装undo-tree.el之后,undo可以用\verb|C-_|,redo可以用\verb|M-_|,想要图形化的操作可以用\verb|C-x u|。
\subparagraph{搜索光标下的字符串}
\verb|M-b|跳到词头后\verb|C-s C-w|,之后\verb|C-s|可循环搜索

\subsection{Work with AUCTEX}
\label{sec:work-with-auctex}
AUCTEX是Emacs上比较好用的\TeX~编辑生成环境,可以很好的自动补全,快速插入章节、命令,也可以很方便的编译、预览等操作。
\subsubsection{编辑}
\label{sec:auctex-edit}
\begin{itemize}
\item 插入section:\verb|C-c C-s|
\item 插入itemize:\verb|C-c C-e|,可选itemize/enumerate/description,写完一个item后,按\verb|C-c C-j|自动创建一个新的item,并对齐。
\item 插入加粗字体:\verb|C-c C-f C-b|
\item 插入斜体字体:\verb|C-c C-f C-i|
\item 插入着重字体:\verb|C-c C-f C-e|
\item 块注释:\verb|C-c %|
\item 折叠文档:\verb|C-c C-o C-b|
\item 反折叠文档:\verb|C-c C-o b|
\end{itemize}

\subsubsection{编译}
\label{sec:auctex-compile}
在\emph{主文件}里面,按\verb|C-c C-c|,即会提示\verb|Command: (default XeLaTeX)|,回车即用此命令编译,稍等一会儿,如果出错,会告诉你敲入\verb|C-c `|查看出错信息;如果没有出错,有中文书签时会提示你需要再运行一次生成书签信息,这时再运行一次\verb|C-c C-c|即可生成中文书签的PDF文件。
\subsubsection{查看书签}
\label{sec:auctex-view-toc}
在主文件里面,按F10可打开终端下的Emacs菜单项,然后选择r,再选择r,即可在左侧打开预览的书签。\verb|C-x o|后跳转到此书签窗口,上下移动高亮条,按空格即可预览该条目信息;按\verb|< / >|键,可调整该条目的层级,功能很强大。其他详细信息,可以按\verb|?|来列出。

\subsection{Helm-mode}
\label{sec:helm-mode}
下载helm-mode的包,解压后放到\verb|~/.emacs.d/site-lisp|下;将下面代码放入\verb|init.el|中,然后\verb|M-x eval-buffer|执行就能生效。\verb|C-x C-f|试试看:
\begin{lstlisting}[language={lisp},label=helm-mode,caption=安装helm-mode]
;; Helm
(add-to-list 'load-path "~/.emacs.d/site-lisp/helm")
(require 'helm-config)
(helm-mode 1)
\end{lstlisting}

\subsection{Work with global}
\label{part:work-with-global}

\subsubsection{安装}
\begin{lstlisting}[language={lisp},label=xgtags-mode,caption=安装xgtags-mode]
;; GNU Global
(add-to-list 'load-path "~/.emacs.d/site-lisp/misc")
(require 'xgtags)
(add-hook 'c-mode-common-hook
          (lambda ()
       (xgtags-mode 1)))
(add-hook 'c++-mode-hook
          (lambda ()
       (xgtags-mode 1)))
\end{lstlisting}

\subsubsection{索引}
在term下,进入项目的根目录,执行\verb|gtags|即可生成索引文件,下一次更新可以用\verb|global -vu|操作。
\subsubsection{操作}
\begin{itemize}
\item 查找tag:\verb|M-.|
\item 跳转下一个tag:\verb|C-c w n|
\item 跳转上一个tag:\verb|C-c w p|
\item 查找symbol:\verb|C-c w s|
\item 查找reference:\verb|C-c w c|
\item 查找文件:\verb|C-c w f|
\end{itemize}


